\documentclass[11pt]{article}
\usepackage{geometry}                % See geometry.pdf to learn the layout options. There are lots.
\geometry{letterpaper}                   % ... or a4paper or a5paper or ... 
%\geometry{landscape}                % Activate for for rotated page geometry
%\usepackage[parfill]{parskip}    % Activate to begin paragraphs with an empty line rather than an indent
\usepackage{graphicx}
\usepackage{amssymb}
\usepackage{epstopdf}
\DeclareGraphicsRule{.tif}{png}{.png}{`convert #1 `dirname #1`/`basename #1 .tif`.png}

\usepackage[english]{babel}
\usepackage[utf8]{inputenc}
\usepackage{fancyhdr}
\pagestyle{fancy}
\fancyhf{}
% \lhead{\begin{tabular}{|p{8.25cm}} \hline {\large \bf SNO+ Laser Procedures} \\ \\ \\ \\ \end{tabular}}
\lhead{\begin{tabular}{|p{8.25cm}|p{6cm}|}  \hline {\large \bf SNO+ Laser Procedures}  
								& Document No:  \\ 
						\cline{2-2} & Revision No: 01\\  
						\cline{2-2} & Effective Date: 2017-04-12\\ 
						\cline{2-2} & Page \thepage \end{tabular}}

\usepackage{hyperref}

\usepackage{geometry}
% \geometry{top=2in}

\begin{document}

\begin{tabular}{||l|l|l||}
\hline\hline
& \multicolumn{2}{p{8cm}||}{\bf SNO+ Laser Procedures} \\
\includegraphics[width=6cm]{snolablogo.pdf} & \multicolumn{2}{p{8cm}||}{} \\
\hline
\multicolumn{2}{||p{8.5cm}|}{Document Number:} & Revision Number: 01\\
\hline
\multicolumn{3}{||l||}{Document Owner: SNO+ Calibration Post-Doc} \\
\hline
\multicolumn{3}{||l||}{Reviewer:}\\
\hline
Name: & Signature & Date \\
\hline
\multicolumn{3}{||l||}{Authorizer:}\\
\hline
Name: & Signature & Date \\
\hline\hline
\end{tabular}
\thispagestyle{empty}
\section{Purpose}

The SNO+ laser is used in conjunction with the laser ball source in the calibration of the detector. Operators must be familiar with safe operation of the laser under standard calibration conditions. The standard procedures associated with this operation are described in this document.

\section{Definitions}
\begin{itemize}
\item {\bf N$_{2}$ laser:} The light source for the calibration which requires a constant flow of N$_{2}$ gas to operate.
\item {\bf Thyratron:} A flat head rectifier that produces the static discharge in nitrogen used that generates the light for the laser.
\item {\bf dye cell:} Curvettes containing the dyes used to generate specific optical wavelengths of light by optical pumping. 
\item {\bf filter wheel:} Two wheels ({\bf a} for course adjustments, {\bf b} for fine adjustments) mounted on stepping motors to select various neutral density filters to adjust the output light intensity.
\item {\bf dye laser:} The extended optics to select a dye cell and guide the resulting light to the optical output aperture.\\ \\ \\
\item {\bf manip:} The program used to run the calibration hardware including the laser.\\ 
\end{itemize}
\section{Commands}
The following are common commands executed through the manip terminal during laser operation.\\ 
\begin{itemize}
\item {\bf n2laser:} Print a list of n2laser commands to the screen.
\item {\bf n2laser monitor:} Displays status information on the n2laser.
\item {\bf n2laser poweron:} Turn on the low voltage control power.
\item {\bf n2laser poweroff:} Turn off the low voltage control power.
\item {\bf n2laser start:} Turn on the high voltage power to the laser.
\item {\bf n2laser stop:} Turn off the high voltage power to the laser.
\item {\bf  dyelaser} Print a list of dye laser commands to the screen.
\item {\bf dyelaser init} Initialize the dye laser.
\item {\bf dyelaser findzero} Find the zero position of the dye laser mirror. The mirror travels down its track till it hits a stop. Must always be executed after {\bf dyelaser init}.
\item {\bf dyelaser cell [0-5]} Select the dye laser cell between 0 and 5.
\item {\bf filterwheel(a$|$b)} Show commands for filter wheel a or b.
\item {\bf filterwheel(a$|$b) init} Initialize the filter wheel.
\item {\bf filterwheel(a$|$b) findtab} Find the calibration tab on filter wheels.
%\item {\bf filterwheelb} Show commands for filter wheel b.
%\item {\bf filterwheelb init} Initialize the filter wheel.
%\item {\bf filterwheelb findtab} Find the calibration tab on filter wheels.bnm

\subsection{Starting the Laser}
\begin{enumerate}
\item \CheckBox[name=sl1]{} Open MV1, the manual valve on the LN$_{2}$ dewar.
\item \CheckBox[name=sl2]{} Open MV2, the pressure builder valve on the LN$_{2}$ dewar.
\item \CheckBox[name=sl3]{} Note the pressure on PG2, the pressure gauge on the dewar. The pressure should be at least 120 psig.
\item \CheckBox[name=sl4]{} Enter DCR and make sure the manual valve, PRV8, is open.
\item \CheckBox[name=sl5]{} Plug in the power cord to the laser
\item \CheckBox[name=sl6]{} On the laser set the POWER switch to {\bf remote}.
\item \CheckBox[name=sl7]{} On the laser set the CONTROL switch to {\bf remote}.
\item \CheckBox[name=sl8]{} Check the manual shutoff valve, MV9, is open.
\item \CheckBox[name=sl9]{} Reset the 'Kill Switch' by pushing the red reset button.
\item \CheckBox[name=sl10]{} Run {\bf n2laser poweron} from the calibration laptop. This will turn on the low voltage control power to the laser and energize the solenoid valve which controls the N$_{2}$ gas flow to the laser head.
\item \CheckBox[name=sl11]{} Note values on \TextField[name=pg2,backgroundcolor=0.975 0.975 0.975,width=1cm]{PG2(}), \TextField[name=pg3,backgroundcolor=0.975 0.975 0.975,width=1cm]{PG3(}), and \TextField[name=fg1,backgroundcolor=0.975 0.975 0.975,width=1cm]{FG1(}). PG2 should be at least 120 psig. PG3 should be at least 100 psig and FG1 should be at least 40 psig. If there is no flow consult and expert. {\bf Running the laser without sufficient N$_{2}$ flow causes serious damage to the laser head!}
\item \CheckBox[name=sl12]{} \TextField[name=stime,backgroundcolor=0.975 0.975 0.975,borderwidth=9]{Note time: [}]
\item \CheckBox[name=sl13]{} At this point the Thyratron should be warming up (controlled by an internal timer) Make sure that the head is thoroughly flushed before turning on the high voltage (wait $\approx$ 10 minutes)
\item \CheckBox[name=sl14]{} Test that the dye cell and neutral density filter selection functions predictably.
\item \CheckBox[name=sl15]{} Use the calibration laptop to read \TextField[name=pt4,backgroundcolor=0.975 0.975 0.975,width=1cm]{PT4(}) and \TextField[name=pt5,backgroundcolor=0.975 0.975 0.975,width=1cm]{PT5(}). PT4 should be at least 80 psi.
\item \CheckBox[name=sl16]{} Make sure the light is blocked with {\bf n2laser block}.
\item \CheckBox[name=sl17]{} Once 10 minutes from the above noted time has elapsed, the thyratron should be warmed up and the head flushed. The laser may then be turned on with the {\bf n2laser start} command.
\end{enumerate}

\subsection{Calibrating the Filter Wheels}
The control systems sometimes lose track of where the filter wheels are positioned. To fix this follow this procedure.
\begin{enumerate}
\item \CheckBox[name=cfw1]{} Reinitialize the filter wheel using the {\bf filterwheela init} command.
\item \CheckBox[name=cfw2]{} Find the tab on the filter wheel using the {\bf filterwheela findtab} command.
\item \CheckBox[name=cfw3]{} Select the desired filter wheel position using the {\bf n2laser setnd} command.
\end{enumerate}
Note that this example was for filter wheel a. Replace {\bf filterwheela} with {\bf filterwheelb} for the second filter wheel.
\subsection{Calibrating the Dye Cell Mirror}
\begin{enumerate}
\item \CheckBox[name=cdm1]{} Reinitialize the mirror {\bf dyelaser init}.
\item \CheckBox[name=cdm2]{} Find the zero of the mirror {\bf dyelaser findzero}.
\item \CheckBox[name=cdm3]{} Select the desired wavelength. Note that the 6 dyecell locations are shared between the 10 logical ones. {\bf dyelaser cell $\bf{<0-9>}$}
\end{enumerate}
\subsection{Shutting Down the Laser}
\begin{enumerate}
\item \CheckBox[name=sdl1]{} Block the light by typing {\bf n2laser block} in the manip terminal.
\item \CheckBox[name=sdl2]{} At the manip terminal type {\bf n2laser stop}.
\item \CheckBox[name=sdl3]{} Type {\bf n2laser poweroff}
\item \CheckBox[name=sdl4]{} Unplug the powercord to the laser.
\item \CheckBox[name=sdl5]{} Turn off the pressure builder (MV2) and then turn off the N$_{2}$ gas (MV1).
\item \CheckBox[name=sdl6]{} Close the manual gas valve on the laser (PRV8).
\end{enumerate}

\section{Operator Notes}
\TextField[name=opd,backgroundcolor=0.975 0.975 0.975,width=4cm]{Operation Date:}\\ 
\TextField[name=opn,backgroundcolor=0.975 0.975 0.975,width=8cm]{Operator Names:}\\ 
Comments:\\
\TextField[name=opt,backgroundcolor=0.975 0.975 0.975,width=15cm,height=4.5cm]{}
\end{Form}
\section{Revision History}
\begin{tabular}{|c|c|c|p{6cm}|}
\hline\hline
\multicolumn{4}{|l|}{Originating Date: 2017-04-12}\\
\hline
Revision No. & Effective Date & Author & Summary of Change \\
& (YYYY-MM-DD) & & \\
\hline
01 & 2017-04-12 & Ryan Bayes & Drafted from SNO Calibration Operators Manual (Laser, Revision 2) with minor revisions and editable blocks added for user notes.\\
\hline
& & & \\
\hline
& & & \\
\hline \hline

\end{tabular}
\end{document}