\documentclass[11pt]{article}
\usepackage{geometry}                % See geometry.pdf to learn the layout options. There are lots.
\geometry{letterpaper}                   % ... or a4paper or a5paper or ... 
%\geometry{landscape}                % Activate for for rotated page geometry
%\usepackage[parfill]{parskip}    % Activate to begin paragraphs with an empty line rather than an indent
\usepackage{graphicx}
\usepackage{amssymb}
\usepackage{epstopdf}
\DeclareGraphicsRule{.tif}{png}{.png}{`convert #1 `dirname #1`/`basename #1 .tif`.png}

\usepackage[english]{babel}
\usepackage[utf8]{inputenc}
\usepackage{fancyhdr}
\pagestyle{fancy}
\fancyhf{}
% \lhead{\begin{tabular}{|p{8.25cm}} \hline {\large \bf SNO+ Laser Procedures} \\ \\ \\ \\ \end{tabular}}
\lhead{\begin{tabular}{|p{8.25cm}|p{6cm}|}  \hline {\large \bf SNO+ Laser Ball PCA Procedures}  
								& Document No:  \\ 
						\cline{2-2} & Revision No: 00\\  
						\cline{2-2} & Effective Date: 2017-04-12\\ 
						\cline{2-2} & Page \thepage \end{tabular}}

\usepackage{hyperref}

\usepackage{geometry}
% \geometry{top=2in}

\begin{document}

\begin{tabular}{||l|l|l||}
\hline\hline
& \multicolumn{2}{p{8cm}||}{\bf SNO+ Laser Procedures} \\
\includegraphics[width=6cm]{snolablogo.pdf} & \multicolumn{2}{p{8cm}||}{} \\
\hline
\multicolumn{2}{||p{8.5cm}|}{Document Number:} & Revision Number: 00\\
\hline
\multicolumn{3}{||l||}{Document Owner: SNO+ Calibration Post-Doc} \\
\hline
\multicolumn{3}{||l||}{Reviewer:}\\
\hline
Name: & Signature & Date \\
\hline
\multicolumn{3}{||l||}{Authorizer:}\\
\hline
Name: & Signature & Date \\
\hline\hline
\end{tabular}
\thispagestyle{empty}

\section{Purpose}

The calibration of the SNO+ PMTs requires a standardized light source. This is provided by a dye laser coupled to the laserball which may be deployed into the detector AV. 

\section{Definitions}

\section{Supporting Procedures}

\section{Commands}

\section{Outline}

The outline of the procedure is:
\begin{enumerate}
\item Prepare the laser and URM for use (turn on N$_{2}$ supply etc).
\item Flush URM2 with N$_{2}$ gas to remove O$_{2}$ and Rn. (if the cover gas system is in place)
\item Calibrate URM2 central rope.
\item Lower source into glovebox.
\item Connect side ropes to source. (if not single axis mode)
\item Deploy source into detector.
\item Take PCA data.
\item Retract source to glovebox.
\item Remove side ropes. (if not single axis mode)
\item Retract source into source tube above gatevalve.
\item Shutdown laser and gas flow to laser and URM.
\end{enumerate} 

\section{Procedures}

\begin{tabular}{|p{8cm}|p{5cm}|}
\hline
&\\
\TextField[name=opd,backgroundcolor=0.975 0.975 0.975,width=4cm]{Operator(s):} &
\TextField[name=opd,backgroundcolor=0.975 0.975 0.975,width=2cm]{Date:} \\
&\\
\hline
\end{tabular}

The procedures in this section are intended to be followed sequentially for the PCA calibration except where it is noted that a following procedure can be skipped. Specifically, if the PCA is to be done in {\it single axis} mode, the side ropes do not need to be attached or detached from the source. 
% Procedures supplementary to the main PCA calibration are found in section 8.3.

\begin{enumerate}
\subsection{Prior to PCA}

\item  \CheckBox[name=prior1]{}  Permission for procedure and confirmation of equipment readiness has been received from Head of Calibration Group or the Site Activities Coordinator. \TextField[name=auth,backgroundcolor=0.975 0.975 0.975,width=3cm]{Record authorizer name}
\item \CheckBox[name=prior2]{} Laserball is mounted in URM2 which is mounted on 10" valve on glovebox.
\item \CheckBox[name=prior3]{} 10" gatevalve is closed and locked.

\subsection{Readying Laser and URM for Operation}

\item \CheckBox[name=rluo1]{} Verify that the LN$_{2}$ dewar in the junction is at least 1/4 full. If not, request that it be swapped out with another dewar. Record liquid level of Dewar, 
\begin{center}
\begin{tabular}{|p{6cm}|}
\hline
\\
\TextField[name=lN2l,backgroundcolor=0.975 0.975 0.975,width=2cm]{LN$_{2}$ Level:} \\
\\
\hline
\end{tabular}
\end{center}
\item \CheckBox[name=rluo2]{} Turn on the N$_{2}$ flow to laser from dewar at junction (marked {\bf Gas Use} on dewar).
\begin{center}
\begin{tabular}{|p{6cm}|}
\hline
\TextField[name=tN2t,backgroundcolor=0.975 0.975 0.975,width=2cm]{Note Time:}\\
\hline
\end{tabular}
\end{center}

\item \CheckBox[name=rluo3]{} Turn on the pressure build valve (marked {\bf Pressure Builder} on dewar). The pressure builder valve opens a controlled leak on the dewar to maintain the 150 psi pressure head. If the valve is not opened, the gas pressure to the laser will eventually drop below the operating level.
\item \CheckBox[name=rluo4]{} Contact the Detector Operator and get permission to enter the DCR. Make sure that the DCR activity bit is set.
\item \CheckBox[name=rluo5]{} Turn on light in DCR following standard procedure. (See Detector Operator Manual).
\item \CheckBox[name=rluo6]{If cover gas present:} Remove the flush return line on the URM. The presence of the buffer line makes it difficult to measure the O$_{2}$ from the URM.
\item \CheckBox[name=rluo7]{If cover gas present:} Check that flush inlet line is connected to URM2. If not connect it. Open the valve on the source tube. It may be necessary to valve off other URMs to get sufficient flow.
\item \CheckBox[name=rluo8]{If cover gas present:} Set up Gas Board in ''bypass mode'' for ''URM flush" only. If you are using the high pressure feed {\bf do not exceed} 10 psi on the regulator. Bypass mode maximizes the flow to the URM.
\item \CheckBox[name=rluo9]{If cover gas present:} Check that the flow meter (located at South-East corner of pipe box) is railed. If not, open needle valve near the flowmeter fully. {\bf Flush should continue until O$_{2}$ reading at the rear of the URM is less than 0.8\%.} This may take up to an hour depending on when the URM was last flushed.
\item \CheckBox[name=rluo10]{} Check that the source clamps are in the OUT position. {\bf Both} knobs have to be in the extreme {\bf OUT} position. {\bf WARNING: If the source is moved with the clamps in the IN position, the source, umbilical, and manipulator may be severely damaged!} The clamps are used to secure the source while the URM is being moved on and off the glovebox.
\item \CheckBox[name=rluo11]{} Check the pressure on the air cylinder for the umbilical takeup mechanism. It should be between 45 and 55 psig. {\bf Do not operate the URM if the pressure is below 40 psig.} If the pressure falls below 10 psig at any point (even momentarily) call the \TextField[name=ocecall,backgroundcolor=0.975 0.975 0.975,width=2cm]{OCE}.
\item \CheckBox[name=rluo12]{} Verify that the 10" gate valve is locked in the closed position. The valve is closed when the handle points towards the pipebox and the slot on the handle stem points AWAY from the source tube.
\item \CheckBox[name=rluo13]{} Calibrate the central rope length (see procedure: Central Rope Position Calibration). Record changes in length of central rop and umbilical. The current fiducial mark for URM2 on the 10" gate valve is $z_{mark} = 1559.9$. Note the fiducial mark is written on the source tube. 
\item \CheckBox[name=rluo14]{} Check that all seals are in place on URM. Including:
	\begin{itemize}
	\item \CheckBox[name=urms1]{} flush inlet line
	\item \CheckBox[name=urms2]{} window on front of URM motorbox
	\item \CheckBox[name=urms3]{} window on rear of URM motorbox
	\item \CheckBox[name=urms4]{} umbilical feedthrouhg on rear of motorbox
	\item \CheckBox[name=urms5]{} view port window cover on source tube
	\item \CheckBox[name=urms6]{} window on rear of strecher box
	\end{itemize}
\item \CheckBox[name=rluo15]{} Check that you are familiar with the section on operating the laser especially the {\bf Emergency Shutdown Procedure}. Also, be aware that UV absorbing safety glasses {\bf MUST} be worn while the laser is running unless {\bf ALL} covers on the laser are in place.
\item \CheckBox[name=rluo16]{} Plug in the powercord to the laser.
\item \CheckBox[name=rluo17]{} Check the POWER switch on the laser is to {\bf local}.
\item \CheckBox[name=rluo18]{} Check the CONTROL switch on the laser is set to {\bf remote}.
\item \CheckBox[name=rluo19]{} Check that the manual shutoff valve on the right of MV5 is open.
\item \CheckBox[name=rluo20]{} Check that the manual shutoff valve MV9 is open.
% \item \CheckBox[name=rluo21]{} Reset the "Kill Switch" by pushing the red reset button.
% \item \CheckBox[name=rluo22]{} Type \verb+n2laser poweron+ on the manip computer. This turns on the low voltage power and energizes the N$_{2}$ gas valve.
\item \CheckBox[name=rluo23]{} Verify that N$_{2}$ is flowing through flow gauge FG5 to the laser head. If there is no flow consult an expert. {\bf Running the laser without sufficient N$_{2}$ flow causes serious damage to the laser head!}
\item \CheckBox[name=rluo24]{} Record observed laser gas pressure and flow values. Note that the expected values listed below may be superseded by ones listed on one or more tags attached to the valves or flow meters. Always use the values found on the tags.
\begin{center}
\begin{tabular}{|l|c|c|}
\hline
Transducer & Expect & Observed \\
\hline
&&\\
PG2 & 110-150 psig & \TextField[name=pg2,backgroundcolor=0.975 0.975 0.975,width=2cm]{}\\
&&\\
\hline
&&\\
PG3 & 90-110 psig & \TextField[name=pg3,backgroundcolor=0.975 0.975 0.975,width=2cm]{} \\
&&\\
\hline
&&\\
PT4 & 100-110 psig & \TextField[name=pt4,backgroundcolor=0.975 0.975 0.975,width=2cm]{} \\
&&\\
\hline
&&\\
FG5 & $\approx 50$ (bottom of ball) & \TextField[name=fg5,backgroundcolor=0.975 0.975 0.975,width=2cm]{} \\
&&\\
\hline
&&\\
PT6 & $\approx 90$ psig & \TextField[name=pg3,backgroundcolor=0.975 0.975 0.975,width=2cm]{} \\
&&\\
\hline
\end{tabular}
\end{center}
PG2, PG3, and FG5 are read off the gas panel on the end of the laser cabinet. PT4 and PT6 are read from the manipulator computer either from the \verb+manmon+ laser window or using the commands,
\begin{center}
\begin{tabular}{ll}
\verb+n2laser hipressure+ & (for PT4) \\
\verb+n2laser lowpressure+ & (for PT6) \\
\end{tabular}
\end{center}
Gas must flow through the laser for $\approx 10$ min before turning the laser high voltage on.
\item \CheckBox[name=rluo25]{} Block the laser light using the command \verb+n2laser block+.
\item \CheckBox[name=rluo26]{} Select dyecell 4 (500 nm) by typing \verb+dyelaser cell 4+
\item \CheckBox[name=rluo27]{} Check the state of the laser by issuing the command \verb+n2laser monitor+. It will tell you the general state of the laser. 
	\begin{itemize}
	\item \CheckBox[name=lst1]{} Check that all 4 stirmotors are on.
	\item \CheckBox[name=lst2]{} Check that there is 120V to the laser.
	\item \CheckBox[name=lst3]{} Check that the filterwheels do not report any problems.
	\end{itemize}
\item \CheckBox[name=rluo28]{If cover gas present} Wait until the O$_2$ level in the URM is at or below 0.8\%.
\subsection{Deploying Source from Source Tube Into Glovebox}
\item \CheckBox[name=rluo29]{If cover gas present:} Verify that the URM is below 0.8\% O$_{2}$
\item \CheckBox[name=rluo30]{If cover gas present:} Check that flush return line is connected to URM2. If not, connect it. It may be necessary to move it from another URM.
\item \CheckBox[name=rluo31]{} Turn off DCR lights.
\item \CheckBox[name=rluo32]{If cover gas present:} Record the Cover Gas O$_{2}$ level.
\begin{center}
\begin{tabular}{|c|}
\hline
\TextField[name=CGO2,backgroundcolor=0.975 0.975 0.975,width=2cm]{Cover Gas O$_{2}$ Reading:}\\
\hline
\end{tabular}
\end{center}
\item \CheckBox[name=rluo33]{} Verify OWL light monitor is on. Establish communications with person watching the light monitor. Suggestion: Station the person watching the OWL monitor at the monitor station. Then he/she can shout through the wall of the DCR and you don't need to use the phones which slow communications down.
\item \CheckBox[name=rluo34]{} Open gate valve ({\bf Slowly!}). Record the time the valve is opened.
\begin{center}
\begin{tabular}{|c|}
\hline
\\
\TextField[name=tgvo,backgroundcolor=0.975 0.975 0.975,width=2cm]{Time Gate Valve Opened:}\\
\\
\hline
\end{tabular}
\end{center}
\item \CheckBox[name=rluo35]{} Lock gate valve open.
\item \CheckBox[name=rluo36]{} With flashlight perform light leak check on URM. In particular check the seal of the source tube window and around the base of the source tube. Also, check around any inspection panel which may have been removed in the recent past.
\item \CheckBox[name=rluo37]{} Using the dimmer switch, {\bf slowly} bring up breaker 9 lights in the DCR with a person still watching the OWL monitor.
\item \CheckBox[name=rluo38]{} In ORCA, source type is set to {\bf LASERBALL}.
\item \CheckBox[name=rluo39]{} ORCA is in a {\bf transitional run}.
\item \CheckBox[name=rluo40]{} Verify that {\bf manip\_logger} on the calibration computer and logging the {\bf Laserball} source.
\item \CheckBox[name=rluo41]{} Check movement of laserball down
\begin{center}
\begin{tabular}{|l|l|}
\hline
console & \verb+ manip > laserball by 0 0 -5+ \\
\hline
manmon & in laserball window: \\ &  set x=0, y=0, z=-5 \\ & click on {\bf move by} \\
\hline
\end{tabular}
\end{center}
The laserball should move down by 5 cm. The tension on the rope should be 40-60 N. The tension on the umbilical should be 10-30 N.
\item \CheckBox[name=rluo42]{} Check that the source offset and orientation is set correctly. At the console type \verb+laserball sourceoffset+. The current laserball has an offset of -64.5 cm. If the reported number is different contact the OCE. FOr single axis deployment mode the orientation should be 0; i.e. confirm from the console by typing \verb+laserball orientation+. It should return \verb+0+. If deployed with sideropes the orientation depends on what direction the slot faces. If in doubt contact the OCE.
\item \CheckBox[name=rluo43]{} Deploy source into the glovebox:
\begin{center}
\begin{tabular}{|l|l|}
\hline
console & \verb+ manip > laserball to 0 0 1380+ \\
\hline
manmon & in laserball window: \\ &  set x=0, y=0, z=1380 \\ & click on {\bf move to} \\
\hline
\end{tabular}
\end{center}

\subsection{Deploying Manipulator into Centre of Detector from Glovebox}
\item \CheckBox[name=rluo44]{} Contact Water Supervisor and advise him/her that the source is being lowered into the AV. The water group maintains a very small differential pressure between the AV water and the cavity water. The volume of the source is enough to disrupt this differential pressure.
\item \CheckBox[name=rlup45]{} Check tensions on urm2rope and urm2umbilical. Rope tension should be appreoximately 30-50 N. Umbilical tension should be between 15-40 N. Note that the tensions are reduced once the sources is submerged.
\item \CheckBox[name=rlup46]{} Move laserball to centre of detector.
\begin{center}
\begin{tabular}{|l|l|}
\hline
console & \verb+ manip > laserball to 0 0 64.5+ \\
\hline
manmon & in laserball window: \\ &  set x=0, y=0, z=64.5 \\ & click on {\bf move to} \\
\hline
\end{tabular}
\end{center}
\subsection{Turning on the Laser}
\item \CheckBox[name=rlup47]{} Verify on the console that the control power on the laser is on: \\ \verb+n2laser monitor+\\ All voltages should be on, all stir motors sould be ON, both filterwheels should be IDLE, the dye cell motor should be IDLE and gas should be flowing. If not contact OCE.
\item \CheckBox[name=rlup48]{} Verify that the N2 gas has been flowing through the laser for $\approx 10$~min. Note that the gas flow is turned on and off with the n2laser poweron/poweroff commands.
\item \CheckBox[name=rlup49]{} Select desired wavelength or dye cell
\begin{center}
\begin{tabular}{|l|l|}
\hline
console & \verb+ manip > dyelaser cell <0-9>+ \\
 & or \\
 & \verb+ manip > dyelaser wavelength <wavelength>+ \\
\hline
manmon & in laser window: \\ & click on button above desired dye cell \\
\hline
\end{tabular}
\end{center}
\item \CheckBox[name=rlup50]{} Set ND=6.0 or higher.
\begin{center}
\begin{tabular}{|l|l|}
\hline
console & \verb+ manip > n2laser setnd 6.0+ \\
\hline
manmon & in laser window: $\to$ Windows $\to$ Neutral Density Settings\\
& click on desired neutral density \\
\hline
\end{tabular}
\end{center}
The filter wheels are set to a large attenuation when first turning on the laser to prevent a large amount of light being introduced into the detecto and overwhelming the data aquisition. Once the laser is on, the rate and intensity can be adjusted to the desired level.
\item  \CheckBox[name=rlup51]{} Wait for the laser to return status READY
\item  \CheckBox[name=rlup52]{} Turn on the laser light
\begin{center}
\begin{tabular}{|l|l|}
\hline
console & \verb+ manip > n2laser start+ \\
\hline
manmon & in laser window: \\
& click on {\bf light on} \\
\hline
\end{tabular}
\end{center}
Now wait 90 seconds while the trigger is delayed. Laser will come on at 10 Hz trigger rate.
\item  \CheckBox[name=rlup52]{} Plug the Laserball Trigger signal into the EXTA input on the MTCD.

\subsection{Taking PCA Data}
\item {\bf PCA Runs} The exact nature of the PCA runs will vary. The "canonical" run tends to be 
	\begin{enumerate}
	\item \CheckBox[name=PCA1]{} Long Low Occupancy Run.
		\begin{itemize}
		\item 500~nm
		\item The centroid for the raw TAC is usually at 1800.
		\item 5-8\% occupancy (ND setting at 500~nm is approximately 3.5).
		\item 40 Hz laser trigger rate (or best you can do without buffer overflow).
		\item Run Type: LASERBALL\_PCA
		\item 10 subruns (2-3 hours)
		\end{itemize}
	\item \CheckBox[name=PCA2]{} Short Medium Occupancy Run.
		\begin{itemize}
		\item 500~nm
		\item The centroid for the raw TAC is usually at 1800.
		\item 20-25\% occupancy (ND setting at 500~nm is approximately 3.0).
		\item 5-10 Hz laser trigger rate (or best you can do without buffer overflow).
		\item Run Type: LASERBALL\_PCA
		\item 3 subruns (20 minjutes).
		\end{itemize}
	\end{enumerate}
	The laser can be run up to 45~Hz but the DAQ produces strange results at such rates.
\subsection{Turning Off Laser}
\item \CheckBox[name=TOL1]{} Unplug EXTA at the MTCD.
\item \CheckBox[name=TOL2]{} Turn off laser light  
\begin{center}
\begin{tabular}{|l|l|}
\hline
console & \verb+ manip > n2laser stop+ \\
\hline
manmon & in laser window: \\
& click on {\bf light off} \\
\hline
\end{tabular}
\end{center}
\item \CheckBox[name=TOL3]{} Turn off laser power
\begin{center}
\begin{tabular}{|l|l|}
\hline
console & \verb+ manip > n2laser poweroff+ \\
\hline
manmon & in laser window: \\
& click on {\bf power off} \\
\hline
\end{tabular}
\end{center}
\item \CheckBox[name=TOL4]{} Unplug laser power cord from wall outlet
The laser is unplugged when it is not intended to be used for extended periods. This is because it has been observed that on several occasions after an unscheduled power outage the laser has come up in a funny state.

\subsection{Retracting Manipulator to Glovebox}
\item \CheckBox[name=rmg1]{} Contact Water Supervisor. Inform him/her that the source is about to be removed from the detector.
\item \CheckBox[name=rmg2]{} Retract the laserball from AV into glovebox
 \begin{center}
\begin{tabular}{|l|l|}
\hline
console & \verb+ manip > laserball to 0 0 1300+ \\
\hline
manmon & in laserball window: \\ &  in laserball window:\\ & click on {\bf Position the pivot} \\ & set x=0, y=0, z=1300 \\ & click on {\bf move to} \\
\hline
\end{tabular}
\end{center}
\item \CheckBox[name=rmg2]{} Retract the laserball to position to disconnect side ropes
 \begin{center}
\begin{tabular}{|l|l|}
\hline
console & \verb+ manip > laserball to 0 0 1380+ \\
\hline
manmon & in laserball window: \\ &  in laserball window:\\ & click on {\bf Position the pivot} \\ & set x=0, y=0, z=1380 \\ & click on {\bf move to} \\
\hline
\end{tabular}
\end{center}
When moving the laserball to 1380, it is important to make sure you are moving with respect to the {\bf pivot} and {\bf not} the centre of the source which is 64.5 cm below the pivot. This is especially important if the sideropes are attached!

\subsection{Retracting source above gate valve; Side ropes not attached}
\item \CheckBox[name=sna1]{} Move laserball to 1530
\begin{center}
\begin{tabular}{|l|l|}
\hline
console & \verb+ manip > laserball to 0 0 1530+ \\
\hline
\end{tabular}
\end{center}
\item \CheckBox[name=sna2]{} Move laserball to 1540
\begin{center}
\begin{tabular}{|l|l|}
\hline
console & \verb+ manip > laserball to 0 0 1540+ \\
\hline
\end{tabular}
\end{center}
\item \CheckBox[name=sna3]{} Move laserball to 1550
\begin{center}
\begin{tabular}{|l|l|}
\hline
console & \verb+ manip > laserball to 0 0 1550+ \\
\hline
\end{tabular}
\end{center}
{\bf NOTE:\\
MINIMUM SAFE HEIGHT TO CLOSE GATEVALVE IS 1535~cm\\
If unable to get above this height, contact expert.}
\item \CheckBox[name=sna4] Retrieve the gatevalve key from the DCR lock box.
\item \CheckBox[name=sna5] Unlock the gatevalve.
\item \CheckBox[name=sna6] Carefully close the gate valve by rotating the handle {\it clockwise}. {\it Expect resistance when the handle is about 3.4 of the way to the closed position. This is the normal overcentering of the valve mechanism.} {\bf If resistance is felt before this or if any sounds are heard that might be caused by valve hitting the source, STOP and contact and expert.} Record the time the valve is closed.
\begin{center}
\begin{tabular}{|c|}
\hline
\\
\TextField[name=tgvc,backgroundcolor=0.975 0.975 0.975,width=2cm]{Time Gate Valve Closed:}\\
\\
\hline
\end{tabular}
\end{center}
\item \CheckBox[name=sna7]{} Lock the gatevalve in the {\bf CLOSED} position.
\item \CheckBox[name=sna8]{} Return the gatevalve key to the DCR lock box.
\item \CheckBox[name=sna9]{If cover gas present} Record the Cover Gas O$_{2}$ level
\begin{center}
\begin{tabular}{|c|}
\hline
\\
\TextField[name=cgor,backgroundcolor=0.975 0.975 0.975,width=2cm]{Cover Gas O$_{2}$ Reading:}\\
\\
\hline
\end{tabular}
\end{center}
\item \CheckBox[name=sna10]{If cover gas present} Close the URM flush valve is the source does not need dryiing out. {\it It is desirable to leave a minute flow of N$_{2}$ through the URM in order to dry out the source and the umbilical. Contact OCE for instructions.}
\item \CheckBox[name=sna11]{If cover gas present} Turn off the URM flush regulator (if the source does not need drying out).
\item \CheckBox[name=sna12]{} IF the laser is off, turn off gas flow at the LN$_{2}$ dewar in the junction:
\subsection{After Calibration}
\item \CheckBox[name=ac1]{} Source is above the gate valve.
\item \CheckBox[name=ac2]{} Gate valve is closed and locked.
\item \CheckBox[name=ac3]{} Laser is off.
\item \CheckBox[name=ac4]{} Manual shutoff valve to the right of MV5 is closed.
\item \CheckBox[name=ac5]{} Laser power cord is unplugged from wall outlet.
\item \CheckBox[name=ac6]{} High pressure LN$_{2}$ dewar is turned off (both {\bf Gas Use} valve and {\bf Pressure Building} valve).
\item \CheckBox[name=ac7]{} Flush return line is disconnected from rear of URM2.
\item \CheckBox[name=ac8]{} Gas board is set up to provide sufficient flow to dry out the inside of the URM.
\end{enumerate}
\end{document}