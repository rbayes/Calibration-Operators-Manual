
% \section{Introduction}



The calibration of the SNO+ detector can be divided into three
aspects:
\begin{enumerate}
\item Calibration of the low level electronics channels (ECA).
\item Calibration of the phototube timing (PCA).
\item Calibration of the detector's physics response.
\end{enumerate}
The low level electronics calibrations (ECA) are done via calibration
circuitry built into the electronics hardware and are not consider
further here. The scope of this document is the equipment and
procedures used to do the phototube calibrations (PCA) and the
multitude of physics calibrations.

The phototube and physics calibration of the SNO+ detector consists of
placing standard sources of light or radioactivity (gamma rays and
electrons) at known locations within the detector. These sources are
used to extract the calibration parameters for both the individual
phototubes (timing resolution, charge vs time, charge resolution,
efficiency) and bulk properties of the detector (wavelength dependent
light attenuation of materials, global detection efficiency).

SNO+ uses known optical and radioactive sources to measure the
calibrations. The optical sources include the \textbf{laserball} and
the \textbf{ELLIE} system while radioactive sources include the
\textbf{$^{16}$N Gamma Ray Source}, the \textbf{$^{8}$Li Cherenkov
  Source}, or the \textbf{$^{270}$Am$^{8}$Be}. Each source has unique
physical properties byt they all share the requirements of being
constructed of materials with little or no radioactivity (except for
the source material in the case of active sources). Specific sources
have been developed for both water and scintillator phases. The
\textbf{Calibration Manipulator System} positions the sources within
the SNO+ detector. The manipulator is a computer controlled system
that is designed to place a source within the acrylic vessel (AV) or
the water between the AV and the PSUP.

\section{Overview of the Calibration Apparatus}

The calibration systems in SNO+ are are situated in the Deck Clean
Room (DCR) in the SNO cavity at SNOLAB. The clean room exists to
minimize the amount of potential background contamination. The central
feature of the DCR is the Universal Interface (UI) to the AV that is
itself the centerpoint of the SNO+ experiment. The UI posseses three
gate valves through which sources may be deployed. Two of these gate
valves are typically occupied by source tubes which partially support
Umbilical Retrieval Mechanisms (URMs) which control the vertical
deployment of calibration sources. The apparatus is designed to
maintain the \textbf{Cover Gas} seal on the detector. Prior to opening
valves to the detector volume, airs is flushed out of the URMs with
pure N$_{2}$ gas derived from liquid nitrogen dewars located in the
Junction. The apparatus also maintains the light seal on the detector
allowing the deployment of sources with the detector turned on.

The URMs use electric motors and a pneumatic
piston to control the movement of a source by applying appropriate
tension on the central rope and umbilical. The URM is instrumented
with encoders and load cells so that the length and the tension on the
umbilical and central rope are known. In the ideal case for a single
axis deployment the tensions are balanced equally between the rope and
the umbilical. Controller boxes attached to the URMs are connected to
a central calibration computer via USB. A lifting mechanism moves the
URMs on and off of the UI and carts are used to move the URMs about
the DCR when they are not on the UI.

Three URMs are in use that were developed for SNO. URM1 was rebuilt to
deploy the Cherenkov source during water phase. Because of the larger
diameter of this source it must be deployed through the 10''
gatevalve. URM2 houses the laserball source and is also mounted on the
10'' gatevalve. Encapulated sources, such as the AmBe source, are
installed on URM2 using the laserball can as a platform, but the
laserball source is restored immediately after deployment. URM3 is the
permanent home of the $^{16}$N gamma ray source and is usually mounted
on the 6'' gatevalve.

For the scintillator phase a new URM1 has been developed with new
materials that are compatible with LAB and better gas systems to
reduce potential background contamination. This system differs
somewhat from the SNO calibration systems in the ways that it
interacts with the cover gas and is mounted on a double gate valve
system so that the URM may be purged more efficiently.

The control of the source positions are supplimented by side
ropes. These ropes are fixed to the inside of the AV and the tensions
are controlled through the use of motors with load cell and encoder
feed back. Sources ride along these ropes on pulleys so that the
relative tension between the central rope and two side ropes guide the
source within a plane. The side ropes are paired to guide the source
in the X-Z plane (using the East and West side ropes) or the Y-Z plane
(using the North and South side ropes). The side ropes are controlled
by the same computer system as the 

Further calibration can be conducted external to the AV. A system of
laser fibers are embedded in on the PSUP for the Embedded LED Laser
Injections Entity (ELLIE). This system 

\section{Contacts}

Below are names and phone numbers for people familiar with different
parts of the calibration system. However, if there is any question
about the state and safety of the system, contact the \textbf{On Call
  Expert} (OCE). Prior to any running of the manipulator the head of
the calibration group or his designate will indicate who is the On
Call Expert.

\section{Rules of (Dis)Engagement}

Chapter 9 of these procedures describe possible problems with the
manipulator system and how to correct them. Some problems are straight
forward and are of little consequence. These problems can be corrected
by the operator. Some problems are significant and the On Call Expert
(OCE) should be contacted before addressing them. The OCE is in turn
required to inform the Head of the Calibration Group of his designate
of any significant problem (here significant is defined as a problem
that could potentially put the manipulator system, the detector or the
scintillator at risk).

{\bf
  \begin{enumerate}
  \item Contact the OCE for any problems that the trouble shooting
    section indicate the OCE should be contacted.
  \item Contact the OCE if you have doubts about a procedure.
  \item Contact the OCE if you have had to execute any of the
    emergency shutdown procedures.
  \item Contact the OCE if any abnormal situation occurs.
  \end{enumerate}


}

