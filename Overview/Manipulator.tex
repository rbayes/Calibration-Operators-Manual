

The SNO+ calibration manipulator is a positioning device used to place
calibration sources inside the AV of the SNO+ detector or down
calibration guide tubes in the region between the AV and PSUP. By
using a system of three ropes; a central rope and two dside ropes, the
manipulator is able to position a source on either an east-west (X-Z) or
north-south (Y-Z) plane inside the AV. About 3/4 of the plane inside
the AV can be reached by the manipulator, while the remaining quarter
is not accessible due to physical limits of the manipulator system
imposed by the geometry. In addition to the manipulator ropes, there
is an umbilical attached to the manipulator that provides the
necessary services for the source, such as electrical signals, fibre
optics or gas lines.

Each calibration source is stored in an Umbilical Retrieval Mechanism
(URM) prior to deployment. A URM consists of a pair of blocks for
taking up the source \textbf{Umbilical} used to provide services to
the source and a \textbf{central rope} used to support the weight of
the source. Below the URM is the \textbf{source tube} which is a 4'
long stainless steel pipe used to store sources when not deployed in
the vessel. Normally, the URM and source tube are mounted on a
calibration port on the \textbf{universal interface} located directly
over the neck of the acrylic vessel. When not in use, the source is
stored in the source tube and a gate valve on the UI seals off the
detector. The central rope in the URM is instrumented with a
\textbf{shaft encoder} which determines the length of rope played out
and a \textbf{load cell} used to measure the tension in the rope. The
umbilical is similarly instrumented.

The layout of the system is shown schematically in
Fig.\ref{fig:manipschema}.

\paragraph{Anchor Blocks}
Each side rope can be thought of as attached at two points (not
exactly true). At the feedthrough where it comes through the UI and at
the \textbf{anchor block} in the AV which is located just above the AV
equator. The end of the rope at the anchor block is fixed abd by
playing the rop in or out through the UI feedthrough, the calibration
source is moved about the AV.

\paragraph{Calibration Guide Tubes}
In addition to deploying sources through the glovebox into the center
of the AV, it is possible to deploy sources through 6 calibration
guide tubes into the cavity water volume between the AV and the
PSUP. These guide tubes are located on the floor of the DCR and are
sealed with gate valves.

\paragraph{Carriage and Weight}
Attached to each calibration source is a carriage and a weight. The
carriage provides attachment points for the central rope and umbilical
and has pullies that the side ropes go around. The weight cylinder is
a stainless steel tube containing lead. The manipulator requires a
miniumum weight for each source to function properly (in particular to
give the sources negative bouyancy) and the wieght cylinder provides
this.

\paragraph{Deck Clean Room(DCR)}
The DCR is the room centered on the deck. Most of the calibration
equipment is located inside the DCR. The DCR is kept clean and has
relatively few airborn particles compared to the rest of the lab. The
clean conditions are maintained to prevent introduction of radioactive
contamination into the detector during the deployment of sources.

\paragraph{Universal Interface (UI)}
The universal interface is a cylindrical box visible in the center of
the DCR through which all of the services into the AV are routed using
a large number of valves and flanges. Sensors, maintained by the
scintillator group to monitor the AV levels, are installed on the UI
alongside the three ports upon which the calibration sources may be
mounted. Additionally, the side rope motor boxes and load cell boxes
are mounted on the UI. The side ropes are attached to the manipulator
carriage of a source using glove ports on the sides of the UI. These
glove ports maintain the darkness within the AV and the integrity of
the radon free cover gas that caps the AV.

\paragraph{Rubbing Ring}
The rubbing ring is an acrylic ring located just below the neck of the
AV inside the target material volume. When the manipulator positions a
source off the central axis, the manipulator ropes and the source
umbilical are pulled to the side of the AV neck. The rubbing rung
provides a wearing surface for the ropes.

\paragraph{Side Rope Motor Mounts}
The spooling mechanisms for the side ropes are mounted on the UI. The
side rope motor boxes are connected to the AV cover gas through VCR
fittings to extension boxes which also conduct the side ropes from the
mechanism to the AV. The motor boxes themselves consist of a motor
driven spool system instrumented with a loadcell to measure the rope
tension and a shaft encoder to measure the rope length. There are four
side ropes, North, South, East and West which are operated in pairs to
allow positioning of the source inside the AV on and East-West or a
North-South plane.

\paragraph{Source Tube}
The stainless steel tube connecting the URM to the calibration
port. The calibration source is parked in the source tube when not
deployed in the detector.

The manipulator carriage and weight assembly is shown schematically in
Fig.~\ref{fig:mancarriage}. It consists of the carriage to which the
central rope is attached. The umbilical passes through the carriage
neck, through the weight assembly into the source. The side ropes are
not attached to the carriage, by rather pass around pullies mounted on
the pully bar. The pully bar and carriage neck are free to rotate
about the pivot. At different positions in the AV the pully bar and
carriage neck will be at different orientations while the weight
assembly and source will always hang vertically below the pivot.

The weigth consists of a stainless steel torus filled with lead. The
lead is potted into the weight cylinder with silicone and then capped
with the cylinder end plate with is sealed with o-rings. Sources are
usually attached to the weight cylinder using the extension tube.

\section{Manipulator Control System}

The manipulator is controlled by the \textbf{calibration} computer
which is a CENTOS Linux PC running a C++ program called manip. The
manip program interacts with the manipulator hardware by controlling a
stepper motor for each axis to change the length of the rope or
umbilical. A shaft encoder on each axis measures the length and a
loadcell measures the tension. The shaft encoders and load cells are
connected to \textbf{controller boxes}, one per two axes. Inside the
controller box is a set of three AVR boards, two to control an axis
(stepper motor, shaft encoder and load cell) and one to concentrate
the data and communicate with the calibration computer via USB. Each
controller box has a unique digital address which is identified as one
of 12 AVR boards. The hub currently in use allows for the use of four
control units at any given time which allows for the four side rope
motor boxes and two URMs to be connected simultaneously. The
N$_{2}$/dye laser is run using the same AVR boards, however it is
connected via an independent USB cable to the calibration computer.

\section{Modes of Source Deployment}

\subsection{Single Axis Deployment}

Sources can be deployed in a single axis mode which consists of
lowering a source straight down from the URM on just the central rope
and umbilical. The horizontal position of the source is determined by
the location of the URM. The vertical position of the source is
determined by the measured length of central rope played out. The
single axis deployment mode is useful for operation along the central
axis of the detector and for deployment of sources down the guide
tubes.

\subsection{Three Axis Deployment}

The main purpose of the manipulator however, is to deploy a source off
the central axis of the detector inside the acrylic vessel. This is
done attaching two side ropes to the manipulator carriage once it is
deployed into the glovebox. The side ropes are attached at one end to
anchor blocks in the AV are anchored at the other end by feedthroughs
on the glovebox. The side ropes go over pullies on the manipulator
carriage. Once the source is lowered into the vessel, it can be pulled
off the central axis by shortening one side rope and lengthening the
other. Because only two side ropes are attached at a time, the source
can only be moved in a plane. There are two sets of side ropes
allowing motion in an east-west plane or a north-south plane. The side
ropes are instrumented in the same fashion as the central rope with
the side rope motor mounts located on the UI. The ropes pass through
steel tubes, thence to extension boxes containing auxiliary pullys and
into the AV. 

