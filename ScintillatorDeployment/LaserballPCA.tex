Note that this is only meant to be a rough sketch of the deployment procedure. Some of the details can only be finalized with the hardware tested in place. 
\begin{enumerate}
\item If the source is not connected to the umbilical; follow steps to connect the laser all to the umbilical source connector which might include
	\begin{enumerate}
  \item Lock the urm into a maintenance position 
	\item Set the urm cover gas to a purge mode
	\item Open the gate valve 
	\item Lower the connector so that it can be manipulated 
	\item Join the laser all to the umbilical with the source connector
	\item Raise the source so that the gate valve can be closed
	\item Continue purging the urm while monitoring radon levels. If using a Rad7 to monitor the levels the reading must be consistent with zero. The purging must be monitored by an operator but the radon levels must meet specification before further activity can occur. 
  \end{enumerate}
\item Connect the urm gate valve to the source cleaning vessel. 
	\begin{enumerate}
  \item Flush nitrogen through the vessel.
	\item Set the urm to purge mode and open the gate valve.
	\item Turn on the spray system and lower and raise the source through the spray jets (might program an automated macro into manip to do this).
	\item Repeated lab samples from the cleaning should be subjected to QA to evaluate contamination over time. Raise the source above the gate valve once criteria is reached (criteria to be established).  
	\item Remove the source cleaning vessel from the urm gate valve. 
  \end{enumerate}
\item Measure the height of the ui relative to the Ibeam. Ensure that the ui height will allow for the source to be used (specification needed). This height must be entered into manip (formula to be added). 
\item With the laser disconnected from the umbilical, start the nitrogen/dye laser. Select the required dye for the desired first wavelength and set the neutral density filters to block the light. Connect the umbilical to the laser. 
\item Unlock the urm from maintenance position and move the urm to mate with the ui gate valve. Clamp the urm over the ui 
\item Connect the urm gate valve to the nipple assembly above the ui gate valve (details to come). 
\item Pump and purge the space between the gate valves using the nipple assembly.
\item While monitoring the pressure in the urm open the urm gate valve to the nipple assembly space. Ensure that the cover gas bag is half inflated (not flat or too full indicating an over or under pressure state). Add or remove nitrogen if needed. 
\item With the knowledge and approval of the shifter, cover gas, and AV operators carefully open the ui gate valve. 
\item With the approval of the cover gas and AV operators, slowly and carefully lower the source into the AV. Monitor the rope and umbilical tensions for changes suggesting that the source meets the lab surface.
\item If using the side ropes lower the source until  it can be reached by the gloves ($\approx$1350 cm). Follow steps to mount the side ropes onto the source carriage. 
\item Lower the source in steps until it is immersed in lab. Then the source can be lowered to the centre of the detector. 
\item Assuming the laser is already running with the neural density filters set to block the light; open the neural density filters to produce the desired intensity per pulse (monitored with EXTA nhits). For a PCA this is between 100 and 200 nhits per pulse. 
\item Follow the run plan
  \begin{enumerate}
	\item For a PCA; 1-2 hours in the centre of the detector such that there are more than 10000 hits per pmt on average. 
	\item For a scan; 15-20 minutes at each location other than the central PCA run. Further details to be set by the run plan. 
  \end{enumerate}
\item When finished; retract the source  with stop points at 1200, 1250, 1300, 1350. Remove the side ropes if necessary. Continue retracting the source in steps reducing in size until the source reaches the reference point. 
\item With the knowledge and approval of the covergas operator, carefully close the ui gate valve. 
\item Check the position of the source pivot through the source tube viewport. Ensure that the pivot is where it is expected to be. 
\item Close the urm gate valve. Drain lab from urm. 
\end{enumerate}
