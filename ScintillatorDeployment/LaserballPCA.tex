The Laserball deployment is meant for the purpose of running PMT calibrations (PCA) and optical measurements from within the scintillator. The following steps must be taken to prepare the source for deployment, operate the dye laser, and deploy the source. 

Note that this is only meant to be a rough sketch of the deployment procedure. Some of the details can only be finalized with the hardware tested in place. 

\section{Source Preparation}

\subsection{Connecting the Source to the Umbilical}
If the source is not connected to the umbilical use the following steps to connect the laserball to the umbilical source connector.
\begin{tabular}{|c|p{4cm}|}
\hline
 &  Lock the urm into a maintenance position  \\ \hline
 &  Set the urm cover gas to a purge mode \\ \hline
 &  Open the gate valve at the end of the URM bellows \\ \hline
 &  Lower the connector so that it can be manipulated \\ \hline
 &  Join the laserball to the umbilical with the source connector \\ \hline
 &  Raise the source so that the gate valve can be closed \\ \hline
 &  Continue purging the urm while monitoring radon levels. If using a Rad7 to monitor the levels the reading must be consistent with zero. The purging must be monitored by an operator while running. The radon levels must meet specification before further activity can occur. If the radon levels do not meet specification put the URM cover gas system in standby mode and resume purging in a subsequent shift. \\ \hline
 \end{tabular}
 
 \subsection{Using the Source Cleaning Vessel}
 The source cleaning vessel must be used prior to deployment to ensure an absence of particulates on the source surface that may contaminate the scintillator in the acrylic vessel. The steps required are as follows;
 \begin{tabular}{|c|p{4cm}|}
 \hline
& Position the genie lift so that the forks will pass through the source cleaning vessel frame lifting points. \\ \hline
& Raise the cleaning vessel frame just enough so that it can be transported to the URM gate valve. \\ \hline
& Check the height of the URM gate valve. If the gate valve is below the SCV flange the position of the gate valve must be adjusted by shortening the length of the turnbuckles supporting the URM gate valve. \\ \hline
& Position the cleaning vessel underneath the URM with the Genie lift wheels parallel to the URM axis. This ensures consistency in the clocking of the flange. \\ \hline
& Lift the SCV up to the URM gate valve. When the SCV is close to the gate valve, start threading the bolts connecting the SCV to the gate valve from underneath. Continue lifting the SCV to the gate valve until the surfaces are in contact. Tighten the bolts to hand tight, then apply no more than a quarter turn on each bolt with a wrench in a star pattern to apply equal force across the flange. Raise the genie lift as much as possible to take the weight of the SCV off of the URM.\\ \hline
& Flush nitrogen through the source cleaning vessel \\ \hline
&  Set the urm to purge mode and open the gate valve. \\ \hline
& Turn on the spray system and lower and raise the source through the spray jets (might program a macro into manip to automate this). \\ \hline
& Repeated lab samples from the cleaning should be subjected to QA to evaluate contamination over time. Raise the source above the gate valve once criteria is reached (criteria to be established).  \\ \hline
& Remove the source cleaning vessel from the urm gate valve by first removing the bolts, lowering the source cleaning vessel on the genie lift, and then removing the source cleaning vessel from underneath the URM. \\ \hline 
  \end{tabular}
  
\section{Calibration Deployment Procedures}

\subsection{Preparation for Deployment}
The following steps must be done before the source deployment. If there is a problem noted here then it is expected that the calibration operation will stop until the problem is resolved with the approval of the site activities coordinator and the calibration expert. 
\begin{tabular}{|c|p{4cm}|}
& Measure the height of the ui relative to the Ibeam. Ensure that the ui height will allow for the source to be used (specification needed). This height must be entered into manip (formula to be added). \\ \hline
& With the laser disconnected from the umbilical, start the nitrogen/dye laser. Select the required dye for the desired first wavelength and set the neutral density filters to block the light. Connect the umbilical to the laser. \\ \hline
& Unlock the urm from maintenance position and move the urm to mate with the ui gate valve. Clamp the urm over the ui \\ \hline
& Connect the urm gate valve to the nipple assembly above the ui gate valve (details to come). \\ \hline
& Pump and purge the space between the gate valves using the nipple assembly. \\ \hline
& While monitoring the pressure in the urm open the urm gate valve to the nipple assembly space. Ensure that the cover gas bag is half inflated (not flat or too full indicating an over or under pressure state). Add or remove nitrogen if needed. \\ \hline
\end{tabular}

\subsection{Deploying the Source}
This is the point of no return for the calibration requiring the approval of the site activities coordinator and the calibration expert. 
\begin{tabular}{|c|p{4cm}|}
& With the knowledge and approval of the shifter, cover gas, and AV operators carefully open the ui gate valve. \\ \hline
& With the approval of the cover gas and AV operators, slowly and carefully lower the source into the AV. Monitor the rope and umbilical tensions for changes suggesting that the source meets the lab surface.\\ \hline
& If using the side ropes lower the source until  it can be reached by the gloves ($\approx$1350 cm). Follow steps to mount the side ropes onto the source carriage. \\ \hline
&  Lower the source in steps until it is immersed in lab. Then the source can be lowered to the centre of the detector. \\ \hline
& Assuming the laser is already running with the neural density filters set to block the light; open the neural density filters to produce the desired intensity per pulse (monitored with EXTA nhits). For a PCA this is between 100 and 200 nhits per pulse. \\ \hline
& Follow the run plan
  \begin{enumerate}
	\item For a PCA; 1-2 hours in the centre of the detector such that there are more than 10000 hits per pmt on average. 
	\item For a scan; 15-20 minutes at each location other than the central PCA run. Further details to be set by the run plan. 
  \end{enumerate} \\ \hline
& When finished; retract the source  with stop points at 1200, 1250, 1300, 1350. Remove the side ropes if necessary. Continue retracting the source in steps reducing in size until the source reaches the reference point. \\ \hline
& With the knowledge and approval of the covergas operator, carefully close the ui gate valve. \\ \hline
& Check the position of the source pivot through the source tube viewport. Ensure that the pivot is where it is expected to be. \\ \hline
& Close the urm gate valve. Drain lab from urm. \\ \hline
\end{tabular}
