\chapter{Source Cleaning Vessel Opearator Manual}
\section{Purpose}
This document is made to compile necessary information concerning the standard procedures that must be followed for the operation and handling of  the source cleaning vessel.
\section{Definition and abbreviation}
\textbf{Cleaning Module}: An apparatus that can be attached directly to the source tube gate valve that sprays a fow of LAB directly onto a source to facilitate cleaning and a drain from which samples of LAB can be drawn for testing.\\
\\
\textbf{Storage Module}: An air tight apparatus in which a source may be stored between deployments when disconnected from the URM.\\
\\
\textbf{Source Cart}:A cart intended only for use in the DCR with a surface suitable for working on sources and a recessed section that can hold the cleaning module. Should have drawers suitable for the storage of tools required for exchanging sources; the necessary tools should stay with the cart.\\
\\
\textbf{Source Connector}:A device by which a source may be temporarily connected to an umbilical while also allowing bre optic or electrical connections.\\
\\
\textbf{URM (Umbilical Retrieval Mechanism)}:The device in which the umbilical is stored
in preparation for deployment that also includes drive mechanisms for the central rope and umbilical to facilitate vertical movement of sources.\\
\\
\textbf{Source Tube}: A solid length of tube connecting to the bottom of the URM with a bellows below that terminates on a gate valve that connects to a nipple assembly on the UI when the URM is in a position ready for deployment.\\
\\
\textbf{Umbilical}: A length of Tygothane tubing containing an optical fiber and five electrical wires to which calibration sources are attached for deployment in the AV.\\
\\
\textbf{Boil off Nitrogen}: Nitrogen sourced from a cryogenic nitrogen dewar that is installed in the Utility drift and is transported to the DCR by way of copper conduit (high pressure) or from the international dewar (low pressure). The pressure of the high pressure nitrogen boil off  for purging purposes must be regulated using the gas board in the DCR to less than 20 psi.








