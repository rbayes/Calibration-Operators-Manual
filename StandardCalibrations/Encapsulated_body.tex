\fancyhf{}
% \lhead{\begin{tabular}{|p{8.25cm}} \hline {\large \bf SNO+ Laser Procedures} \\ \\ \\ \\ \end{tabular}}
\lhead{\begin{tabular}{|p{8.25cm}|p{6cm}|}  
	\hline 
		{\large \bf Acrylic Calibration Procedure}  
										& Document No:  \\ 
								\cline{2-2} & Revision No: 01\\  
								\cline{2-2} & Effective Date: 2017-08-25\\ 
								\cline{2-2} & Page \thepage \\  \end{tabular}}



\begin{tabular}{||l|l|l||}
\hline\hline
& \multicolumn{2}{p{8cm}||}{\bf Encapsulated Calibration Procedure} \\
\includegraphics[width=6cm]{figures/SNOplus_logo.png} & \multicolumn{2}{p{8cm}||}{} \\
\hline
\multicolumn{2}{||p{8.5cm}|}{Document Number:} & Revision Number: 01\\
\hline
\multicolumn{3}{||l||}{Document Owner: SNO+ Calibration Post-Doc} \\
\hline
\multicolumn{3}{||l||}{Reviewer:}\\
\hline
Name: & Signature & Date \\
\hline
\multicolumn{3}{||l||}{Authorizer:}\\
\hline
Name: & Signature & Date \\
\hline\hline
\end{tabular}
\thispagestyle{empty}

\section{Purpose}

The procedure describes step by strp the deployment process for an
acryilic source into the AV. It is basically identical to the
Laserball or N16 deploment procedure. It is intendedd as a guideline
for operators who have been trained on the manipulator and laser. It
assumes that the laserball is mounted on URM2 which is located on the
10'' gatevalve on the UI.

The outline of the procedure is:
\begin{enumerate}
\item Prepare the URM for use (turn on N$_{2}$ supply etc).
\item Flush URM2 with N$_{2}$ gas to remove O$_{2}$ and Rn.
\item Calibrate URM2 central rope.
\item Lower source into glovebox.
\item Conneect side ropes to source (if not single axis mode).
\item Deploy source into detector.
\item Take data.
\item Retract source into UI.
\item Remove side ropes (if not single axis mode).
\item Retract source inot source tube above gatevalve.
\item Shut down gas flow to URM.
\end{enumerate}

% \begin{form}

\begin{tabular}{|l|l|}
\hline
\multicolumn{2}{|l|}{} \\
\multicolumn{2}{|l|}{\bf Encapsulated Calibration Procedure} \\
\multicolumn{2}{|l|}{} \\
\hline
& \\
\TextField[name=acalop,backgroundcolor=0.975 0.975 0.975,width=2cm]{Operator: } &
\TextField[name=acald,backgroundcolor=0.975 0.975 0.975,width=4cm]{Date: } \\
& \\
\hline
\end{tabular}

The procedures in this section are intended to be followed
sequentially fro the source calibration except where it is noted the a
following procedure can be skipped. Specifically if the source run is
to be done in {\it single axis} mode, the side ropes do not need to be
attached or detached from the source.


{\bf Prior to Calibration Run}
\begin{enumerate}
  
\item \CheckBox[name=encp1]{} Permission for procedure and confirmation of equipment rediness
  has been received from Head of Calibration Group.
  
  \begin{center}
    \begin{tabular}{|l|}
      \hline
      \\
      \TextField[name=acalHCG,backgroundcolor=0.975 0.975
        0.975,width=2cm]{Authorizor's name:}
      \\
      \hline
    \end{tabular}
  \end{center}
  
\item \CheckBox[name=encp2]{} Source is mounted in URM2 which is mounted on 10'' valve on UI.
\item \CheckBox[name=encp3]{} 10'' gatevalve is closed and locked.

  {\bf Readying URM from Operation}

  \item \CheckBox[name=encp4]{} Verify that the LN$_{2}$ dewar in the junction is at least 1/4
    full. If not sway is out with another dewar. Record the liquid
    level of the dewar.
    
    \begin{center}
      \begin{tabular}{|l|}
        \hline
        \\
        \TextField[name=ln2level,backgroundcolor=0.975 0.975
          0.975,width=2cm]{LN$_{2}$ Level:}\\
        \\
        \hline
      \end{tabular}
    \end{center}
    
\item \CheckBox[name=encp5]{} Verify that the dewar gas pressure is approximately 130 to 150
  psig. If not, swap it out with another dewar.
\item \CheckBox[name=encp6]{} Turn on N$_{2}$ flow to laser from dewar at junction.
  
  \begin{center}
    \begin{tabular}{|l|}
      \hline
      \\
      \TextField[name=n2time,backgroundcolor=0.975 0.975 0.975,
        width=2cm]{Note Time:}\\
      \\
    \end{tabular}
  \end{center}

\item \CheckBox[name=encp7]{} Turn on pressure builder valve. {\it the pressure builder opens
  a controlled leak on the dewar to maintain the 150 psi pressure
  head. If the valve is not opened, the gas pressure to the laser
  will eventually drop below the operating level.}
\item \CheckBox[name=encp8]{} Once the URM is flushed, the N$_{2}$ supply may be switched
  from the high pressure dewar to the Wessington dewar. Check with
  the Operations Group first before switching. Do not use the
  Wessington if a transfer is in progress. Consult the gasboard
  section for details on how to switch.
\item \CheckBox[name=encp9]{} Contact Detector Operator and get permission to enter DCR. Make
  sure that the DCR activity bit is set.
\item \CheckBox[name=encp10]{} Turn on lights in DCR following standard procedure. (See
  Detector Operator Manual)
\item \CheckBox[name=encp1]{} Remove the flush return line on the URM. {\it The presence of the
  buffer line makes it difficult to measure the O$_{2}$ from the URM.}
\item \CheckBox[name=encp11]{} Check that the flush inlet line is connected to URM2. If not
  connect it. Open the valve on the source tube. {\it It may be
    necessary to valve off other URMs to get sufficient flow.}
\item \CheckBox[name=encp12]{} Set up Gas Board in `bypass mode' for `URM flush' only. If you
  are using the high pressure feed {\bf do not exceed} 10 psi on the
  regulator. {\it Bypass mode maximizes the flow to the URM.}
\item \CheckBox[name=encp13]{} Check that the flow meter (located at the South-East corner of
  the pipe box) is railed. If not, open needle valve near the
  flowmeter fully. {\bf Flush should continue until O$_2$ reading at
    the rear of the URM is less than 0.8\%.} {\it This may take up to
    an hour depending on when the URM was last flushed.}
\item \CheckBox[name=encp14]{} Check that the source clamps are in the OUT position. {\bf Both}
  knobs have to be in the extreme {\bf OUT} position. {\bf WARNING: If
    the source is moved with the clamps in the IN position, the
    source, umbilical and manipulator may be severely damaged!} {\it
    The clamps are used to secure the source while the URM is being
    moved on and off the glovebox.}
\item \CheckBox[name=encp15]{} Check the pressure on the air cylinder for the umbilical takeup
  mechanism. It should be between 45 and 55 psig. {\bf DO not operate
    the URM if the pressue is below 40 psig}. If the pressure falls
  below 10 psig at any point (even momentarily) call the OCE. An
  internal inspection of the URM is mandatory before operating the
  unig again. {\it The pressure cylinder on the URM maintains tension
    on the umbilical takeup reel. A low gas pressure can result in the
    umbilical falling off the takeup reel and getting caught or jammed
    leading to destruction of the umbilical.}
\item \CheckBox[name=encp16]{} Verify that the 10'' gatevalve is locked in the closed
  position. {\it The valve is CLOSED when the hangle points in toward
    the pipes and the slot on the handle stem points AWAY from the
    source tube.}
\item \CheckBox[name=encp17]{} Calibrate Central Rope Length (see procedure). Record changes in
  length of central rope and umbilical. The current fiducial mark for
  URM2 on the 10'' gatevalve is
  \[
  z_{mark} = 1549.3
  \]
  Note: the fiducial mark is written on the source tube. If it
  differes from the above number use it instead.

  \begin{center}
    \begin{tabular}{l}
      \hline
      \\
      \TextField[name=crcal1,backgroundcolor=0.975 0.975 0.975,
        width=2cm]{$\Delta$l rope:} \\
      \\
      \hline
      \\
      \TextField[name=cucal1,backgroundcolor=0.975 0.975 0.975,
        width=2cm]{$\Delta$l rope:} \\
      \\
      \hline
    \end{tabular}
  \end{center}

\item \CheckBox[name=encp18]{} Check that all seals are in place on URM. Including:
  \begin{itemize}
  \item \CheckBox[name=encp19]{} flush inlet line
  \item \CheckBox[name=encp20]{} window on front of URM motorbox
  \item \CheckBox[name=encp21]{} window on rear of URM motorbox
  \item \CheckBox[name=encp22]{} umbilical feedthrough on rear of motorbox
  \item \CheckBox[name=encp23]{} view port window cover on source tube
  \item \CheckBox[name=encp24]{} window on rear of stretcher box
  \end{itemize}
\item \CheckBox[name=encp25]{} Wait until the O$_{2}$ level in the URM is at or below 0.8\%

  \begin{center}
    {\bf Deploying Source rfom Source Tube Into Glovebox}
  \end{center}
  
\item \CheckBox[name=encp26]{} Verify that the URM is below 0.8\% O$_{2}$.
\item \CheckBox[name=encp27]{} Check that flush return line is connected to URM2. If not
  connect it. {\it It may be necessary to move it from another URM.}
\item \CheckBox[name=encp28]{} Turn off DCR lights.
\item \CheckBox[name=encp29]{} Record the Cover Gas O$_{2}$ level
  \begin{center}
    \begin{tabular}{l}
      \hline
      \\
      \TextField[name=cg0ra,backgroundcolor=0.975 0.975 0.975,
        width=2cm]{Cover Gas O$_{2}$ Reading:}\\
      \\
      \hline
    \end{tabular}
  \end{center}
\item \CheckBox[name=encp30]{} Establish communication with the detector operator. Ensure that
  the detector is in a Transition run. The detector operator should
  monitor the detector rates.
\item \CheckBox[name=encp31]{} Open gate valve ({\bf Slowly!}.\\
  Record the time the valve is opened.
  \begin{center}
    \begin{tabular}{|l|}
      \hline \\ \TextField[name=tgvoa,backgroundcolor=0.975 0.975 0.975,
        width=2cm]{Time Gate Valve Opened}\\
      \\
      \hline
    \end{tabular}
  \end{center}

\item \CheckBox[name=encp32]{} Secure the gate valve either by locking or removing the handle.
\item \CheckBox[name=encp33]{} With flashlight, and participation of the detector operator,
  perform a light leak check on the URM. In particular, check the seal
  of the source tube window and around the base of the source
  tube. Also, check around any inspection panel which may have been
  removed in the recent past.
\item \CheckBox[name=encp34]{} Bring up the lights on breaker 9 in the DCR with the detector
  operator monitoring the detector rate.
\item \CheckBox[name=encp35]{} Verify that {\bf manip\_logger} is running on the calibration
  computer and is logging the Encapsulated source by checking the latest
  entries in couch.ug.snopl.us/manip/\_all\_docs.
\item \CheckBox[name=encp36]{} Check movement of the acrylic source down:
  \begin{center}
    \begin{tabular}{|l|l|}
      \hline
      console & \verb+manip > ambe by 0 0 -5+ \\
      \hline
      manmon & in ambe window: \\
      & set x=0, y=0, z=-5\\
      & click on {\bf move by} \\
      \hline
    \end{tabular}
  \end{center}
  {\it The source should move down 5 cm. The tension on the rope
    should be 40-60 N. The tension on the umbilical should be 10-30
    N.}
\item \CheckBox[name=encp37]{} Check that the source offset is set correctly. At the console
  type \verb+acrylic sourceoffset+. The actual offset depends on the
  exact configuration (canned/un-canned, spacer present etc.)
\item \CheckBox[name=encp38]{} Deploy source into the glovebox:
  \begin{center}
    \begin{tabular}{|l|l|}
      \hline
      console & \verb+manip > ambe to 0 0 1380+ \\
      \hline
      manmon & in ambe window: \\
      & set x=0, y=0, z=1380\\
      & click on {\bf move to} \\
      \hline
    \end{tabular}
  \end{center}
\item \CheckBox[name=encp39]{} Contact Water Supervisor and advise him/her that the source is
  being lowered into the D$_{2}$O. {\it The water group maintains a
    very small differential pressure between the water inside and
    outside of the heavy water. The volume of the source is enough to
    disrupt this differential pressure.}
\item \CheckBox[name=encp40]{} Check tension of urm2rope and urm2umbilical. Rope tension should
  be approximately 30-50 N. Umbilical tension should be between 15-40
  N. Note that the tensions are reduced once the source is submerged.
\item \CheckBox[name=encp41]{} Move acrylic source to centre of detector (assuming source
  offset is 70 cm).
  \begin{center}
    \begin{tabular}{|l|l|}
      \hline
      console & \verb+manip > acrylic to 0 0 70+ \\
      \hline
      manmon & in acrylic window: \\
      & set x=0, y=0, z=70\\
      & click on {\bf move to} \\
      \hline
    \end{tabular}
  \end{center}      
\item \CheckBox[name=encp42]{} Take data. The exact configuration will vary.

  \begin{center} {\bf Retracting Manipulator to UI} \end{center}

\item \CheckBox[name=encp43]{} Contact Water Supervisor. Inform him/her that the source is
  about to be removed from the D$_{2}$O.
\item \CheckBox[name=encp44]{} Retract encapsulated source from AV into glovebox.
  \begin{center}
    \begin{tabular}{|l|l|}
      \hline
      console & \verb+manip > ambe to 0 0 1300+ \\
      \hline
      manmon & in ambe window: \\
      & set x=0, y=0, z=1300\\
      & click on {\bf move to} \\
      \hline
    \end{tabular}
  \end{center}
\item \CheckBox[name=encp44]{} Retract encapsulated source to posisition to disconnect side ropes
  \begin{center}
    \begin{tabular}{|l|l|}
      \hline
      console & \verb+manip > ambe to 0 0 1380+ \\
      \hline
      manmon & in ambe window: \\
      & set x=0, y=0, z=1380\\
      & click on {\bf move to} \\
      \hline
    \end{tabular}
  \end{center}
  {\it When moving the encapsulated source to 1380, it is important to make
    sure you are moving with respect to the {\bf pivot} and not the
    centre for the source which is approximately 70 cm below the
    pivot. This is especially important if the sideropes are
    attahced!}
  
  \begin{center} {\bf Retracting source above gate valve. Side ropes
      NOT attached.}
  \end{center}
  
\item \CheckBox[name=encp45]{} Move the encapsulated source to 1530
  \begin{center}
    \begin{tabular}{|l|l|}
      \hline
      console & \verb+manip > ambe to 0 0 1530+ \\
      \hline
    \end{tabular}
  \end{center}
\item \CheckBox[name=encp46]{} Move the encapsulated source to 1540
  \begin{center}
    \begin{tabular}{|l|l|}
      \hline
      console & \verb+manip > ambe to 0 0 1540+ \\
      \hline
    \end{tabular}
  \end{center}
\item \CheckBox[name=encp47]{} Move the encapsulated source to 1550
  \begin{center}
    \begin{tabular}{|l|l|}
      \hline
      console & \verb+manip > ambe to 0 0 1550+ \\
      \hline
    \end{tabular}
  \end{center}
      {\bf NOTE:\\
        MINIMUM SAFE HEIGHT TO CLOSE GATEVALVE IS 1535~cm.\\
        If unable to get above this height, contact expert.}
    \item \CheckBox[name=encp48]{} Retrieve the gatevalve key from the DCR lock box, if the
      gate valve handle is locked. Retrieve the handle from the DCR
      desk if the handle was removed.
    \item \CheckBox[name=encp49]{} Ready the gatevalve to be closed by unlocking or replacing
      the handle.
    \item \CheckBox[name=encp50]{} Carefully close the gate valve by rotating the handle {\bf
      clockwise}. {\it Expect resistance when the handle is about 3.4
      of the way to the closed position. This is the normal
      overcentering of the valve mechanism.} {\bf If resistance is
      felt before this or if any sounds are heard that might be caused
      by the valve hitting the source, STOP and contact and expert.}
      Record the time the valve is closed.

      \begin{center}
        \begin{tabular}{|l|}
          \hline \\ \TextField[name=tgvca,backgroundcolor=0.975 0.975 0.975,
        width=2cm]{Time Gate Valve Closed}\\
          \\
          \hline
        \end{tabular}
      \end{center}

    \item \CheckBox[name=encp51]{} Lock the gatevalve in the {\bf CLOSED} position.
    \item \CheckBox[name=encp52]{} Return the gatevalve key to the DCR lock box.
    \item \CheckBox[name=encp53]{} Record the Cover Gas O$_{2}$ level
      
      \begin{center}
        \begin{tabular}{|l|}
          \hline \\ \TextField[name=cgora1,backgroundcolor=0.975 0.975 0.975,
        width=2cm]{Time Gate Valve Closed}\\
          \\
          \hline
        \end{tabular}
      \end{center}

    \item \CheckBox[name=encp54]{} Close the URM flush valve if the source does not need drying
      out. {\it It is desireable to leave a minute flow of N$_{2}$
        through the URM in order to dry out the source and the
        umbilical. Contact OCE for instruction.}
    \item \CheckBox[name=encp55]{} Turn off the URM flush regulator (if the source does not
      need drying out).
    \item \CheckBox[name=encp56]{} If the laser is off, turn off the gas flow at the LN$_{2}$
      dear in the junction:
      \begin{enumerate}
      \item \CheckBox[name=encp57]{} close the {\bf Gas Use} valve.
      \item \CheckBox[name=encp1]{} close the {\bf Pressure Building} valve.
      \end{enumerate}
    \item \CheckBox[name=encp58]{} If the source is retracted, and gate valve closed, turn off
      gas flow at the high pressure LN$_{2}$ dewar in the junction:
      \begin{enumerate}
      \item \CheckBox[name=encp59]{} close the {\bf Gas Use} valve.
      \item \CheckBox[name=encp60]{} close the {\bf Pressure Building} valve.
      \end{enumerate}
      \begin{center} {\bf Afterr Calibration} \end{center}
    \item \CheckBox[name=encp61]{} Source is above gate valve.
    \item \CheckBox[name=encp62]{} Gate valve is closed and locked.
    \item \CheckBox[name=encp63]{} High pressure LN$_{2}$ dewar is turned off (bot {\bf Gas
      Use} valve and {\bf Pressure Building} valve) if source is not
      to be dryed out.
    \item \CheckBox[name=encp64]{} Flush return line is disconnected from rear of URM2.
    \item \CheckBox[name=encp65]{} Gas board is set up to provide sufficient flow to dry out
      the inside of the URM.
\end{enumerate}

% \end{form}

